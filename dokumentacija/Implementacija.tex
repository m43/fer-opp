\chapter{Implementacija i korisničko sučelje}
		
		
		\section{Korištene tehnologije i alati}
			
			\textnormal{
			Timski rad na ovom projektu zahtjeva konstanto surađivanje i praćenje
			stupnja razvijenosti aplikacije u procesu razvoja. Stoga je kao sustav za praćenje razvoja aplikacije korišten git. Git se sastoji od udaljenog, zajedničkog repozitorija, pohranjenog na poslužitelju GitLab, i lokalnih repozitorija svakog timskog člana što osigurava kontinuirano praćenje napretka, te na taj način svi imaju uvid u najnovije stanje i promjene dokumentacije i izvornoga koda aplikacije.}
		
			\smallbreak
			
			\textnormal{
			Za izradu dokumentacije korišten je programski jezik LaTeX. Za
			razliku od ostalih poznatih uređivača teksta LaTeX dopušta surađivanje sustavom Git. Dokumentacija je pisana u uređivačkom programu TeXstudio, a odabrani prevoditelj je texLive. Dijagrami su napravljeni u uređivačkom programu Astah Professional, a u dokumentaciju su uvezeni kao .png slike generirane is Astah programa.}
		
			\smallbreak
			
			\textnormal{
			Za razvojno okruženje odabran je Microsoft Visual Studio, tvrtke
			Microsoft. Za radni okvir aplikacije odabran je ASP.NET Core koji podržava razvoj web aplikacija uključujući web servise, web resurse i web API-je. Razvojni model aplikacije je MVC (Model, View, Controller).
			Datoteke View komponente (frontend) su pisane u prezentacijskom jeziku HTML, stilskom jeziku CSS i skriptnom programskom jeziku JavaScript.
			Za kreiranje reaktivnih stranica (single-page applications, SPA) i korisničkih sučelja u frontendu korišten je Vue.js javascript okvir.
			Za pisanje datoteka u komponentama Model i Controller (backend) korišten je programski jezik C\#.
			Baza podataka se nalazi na poslužitelju u oblaku Microsoft Azure.\\
			\\
			\url{https://git-scm.com/}\\
			\url{https://gitlab.com/}\\
			\url{https://www.latex-project.org/}\\
			\url{http://astah.net/editions/professional}\\
			\url{https://visualstudio.microsoft.com/}\\
			\url{https://docs.microsoft.com/en-us/aspnet/core/?view=aspnetcore-3.1}\\
			\url{https://docs.microsoft.com/en-us/dotnet/csharp/}\\
			\url{https://www.javascript.com/}\\
			\url{https://portal.azure.com/}\\
			\url{https://vuejs.org/}}
			
			
			\eject 
		
	
		\section{Ispitivanje programskog rješenja}
			
			\textbf{\textit{dio 2. revizije}}\\
			
			 \textit{U ovom poglavlju je potrebno opisati provedbu ispitivanja implementiranih funkcionalnosti na razini komponenti i na razini cijelog sustava s prikazom odabranih ispitnih slučajeva. Studenti trebaju ispitati temeljnu funkcionalnost i rubne uvjete.}
	
			
			\subsection{Ispitivanje komponenti}
			\textit{Potrebno je provesti ispitivanje jedinica (engl. unit testing) nad razredima koji implementiraju temeljne funkcionalnosti. Razraditi \textbf{minimalno 6 ispitnih slučajeva} u kojima će se ispitati redovni slučajevi, rubni uvjeti te izazivanje pogreške (engl. exception throwing). Poželjno je stvoriti i ispitni slučaj koji koristi funkcionalnosti koje nisu implementirane. Potrebno je priložiti izvorni kôd svih ispitnih slučajeva te prikaz rezultata izvođenja ispita u razvojnom okruženju (prolaz/pad ispita). }
			
			
			
			\subsection{Ispitivanje sustava}
			
			 \textit{Potrebno je provesti i opisati ispitivanje sustava koristeći radni okvir Selenium\footnote{\url{https://www.seleniumhq.org/}}. Razraditi \textbf{minimalno 4 ispitna slučaja} u kojima će se ispitati redovni slučajevi, rubni uvjeti te poziv funkcionalnosti koja nije implementirana/izaziva pogrešku kako bi se vidjelo na koji način sustav reagira kada nešto nije u potpunosti ostvareno. Ispitni slučaj se treba sastojati od ulaza (npr. korisničko ime i lozinka), očekivanog izlaza ili rezultata, koraka ispitivanja i dobivenog izlaza ili rezultata.\\ }
			 
			 \textit{Izradu ispitnih slučajeva pomoću radnog okvira Selenium moguće je provesti pomoću jednog od sljedeća dva alata:}
			 \begin{itemize}
			 	\item \textit{dodatak za preglednik \textbf{Selenium IDE} - snimanje korisnikovih akcija radi automatskog ponavljanja ispita	}
			 	\item \textit{\textbf{Selenium WebDriver} - podrška za pisanje ispita u jezicima Java, C\#, PHP koristeći posebno programsko sučelje.}
			 \end{itemize}
		 	\textit{Detalji o korištenju alata Selenium bit će prikazani na posebnom predavanju tijekom semestra.}
			
			\eject 
		
		
		\section{Dijagram razmještaja}
			
			\textbf{\textit{dio 2. revizije}}
			
			 \textit{Potrebno je umetnuti \textbf{specifikacijski} dijagram razmještaja i opisati ga. Moguće je umjesto specifikacijskog dijagrama razmještaja umetnuti dijagram razmještaja instanci, pod uvjetom da taj dijagram bolje opisuje neki važniji dio sustava.}
			
			\eject 
		
		\section{Upute za puštanje u pogon}
		
			\textbf{\textit{dio 2. revizije}}\\
		
			 \textit{U ovom poglavlju potrebno je dati upute za puštanje u pogon (engl. deployment) ostvarene aplikacije. Na primjer, za web aplikacije, opisati postupak kojim se od izvornog kôda dolazi do potpuno postavljene baze podataka i poslužitelja koji odgovara na upite korisnika. Za mobilnu aplikaciju, postupak kojim se aplikacija izgradi, te postavi na neku od trgovina. Za stolnu (engl. desktop) aplikaciju, postupak kojim se aplikacija instalira na računalo. Ukoliko mobilne i stolne aplikacije komuniciraju s poslužiteljem i/ili bazom podataka, opisati i postupak njihovog postavljanja. Pri izradi uputa preporučuje se \textbf{naglasiti korake instalacije uporabom natuknica} te koristiti što je više moguće \textbf{slike ekrana} (engl. screenshots) kako bi upute bile jasne i jednostavne za slijediti.}
			
			
			 \textit{Dovršenu aplikaciju potrebno je pokrenuti na javno dostupnom poslužitelju. Studentima se preporuča korištenje neke od sljedećih besplatnih usluga: \href{https://aws.amazon.com/}{Amazon AWS}, \href{https://azure.microsoft.com/en-us/}{Microsoft Azure} ili \href{https://www.heroku.com/}{Heroku}. Mobilne aplikacije trebaju biti objavljene na F-Droid, Google Play ili Amazon App trgovini.}
			
			
			\eject 