\chapter{Arhitektura i dizajn sustava}
		

		\textnormal{ Arhitektura sustava ovog projekta može se podijeliti u tri sustava:}
	\begin{itemize}
		\item 	\textit{Baza podataka}
		\item 	\textit{Web aplikacija}
		\item 	\textit{Web poslužitelj}		
	\end{itemize}
\textnormal {\textbf{Web poslužitelj} osnova je rada web aplikacije. Njegova primarna zadaća je komunikacija klijenta s aplikacijom. Komunikacija se odvija preko HTTP (engl. Hyper Text Transfer Protocol) protokola. Poslužitelj je onaj koji pokreće web aplikaciju te joj prosljeđuje zahtjev. \textbf{Web preglednik} je program koji korisniku omogućuje pregled web stranica i multimedijalnih sadržaja vezanih uz njih.Korisnik i web aplikacija komuniciraju s web preglednikom. Korisnik preko \textbf{web aplikacije} obrađuje zahtjeve. Web aplikacija obrađuje zahtjeve te komunicira s bazom podataka. }
\bigbreak
\textnormal {Programski jezik koji smo odabrali za našu web aplikaciju jest C\# Odabrali smo ga jer je većina današnjih web aplikacija napisana u C\#. Također, odlučili smo se na ASP NET CORE 3 radni okvir koji pruža mnogo funkcionalnosti. Koristimo i Vue.js radni okvir za JavaScript. Za razvojno okruženje odabrali smo Microsoft Visual Studio.}
\bigbreak
\textnormal{Arhitektura sustava temelji se na MVC(Model-View-Controller) konceptu.Taj je koncep podržan od ASP NET CORE 3 radnog okvira i olakšava razvoj web aplikacije zbog svojih gotovih predložaka. MVC koncept karakterističan je po tome što se pojedini dijelovi aplikacije mogu razvijati nezavisno, što uvelike olakšava testiranje, razvijanje te dodavanje novih mogućnosti u sustav. MVC koncep sadržava:}
\bigbreak
	\begin{itemize}
	\item 	\textbf{Model} - prima ulazne podatke od Controllera, predstavlja strukture podataka koje su dinamične i neovisne o korisničkom sučelju. Središnja je komponenta sustava jer izravno upravlja pravilima web aplikacije.
	\item 	\textbf{View} - prikaz podataka s mogućnosću različitih prikaza.
	\item 	\textbf{Controller}	- upravlja korisničkim zahtjevima te prima ulaze i prilagođava ih za prosljeđivanje Modelu ili Viewu, ovisno o potrebi.
\end{itemize}

	
		

		

				
		\section{Baza podataka}
			
			
		\textnormal{Za naš projekt odabrali smo relacijsku bazu podataka. Njezina osnovna jedinica je tablica definirana imenom i skupom atributa. Zadaća naše baze podataka je brza i jednostavna pohrana podataka, izmjena, brisanje i dohvat tih podataka. Baza podataka ove aplikacije sastoji se od sljedećih entiteta:}
		\bigbreak
			\begin{itemize}
			\item  Košarica
			\item  Korisnik
			\item  Narudžba
			\item Transakcija
			\item Artikl košarice
			\item Artikl narudžbe
			\item Artikl
			\item Artikl dostupnost
			\item Recenzija
			\item Utakmica
			\item Popust
			\item Slika
			\item Objava
			\item Igrač
		\end{itemize}
		
		
			\subsection{Opis tablica}
			

				\textit{Svaku tablicu je potrebno opisati po zadanom predlošku. Lijevo se nalazi točno ime varijable u bazi podataka, u sredini se nalazi tip podataka, a desno se nalazi opis varijable. Svjetlozelenom bojom označite primarni ključ. Svjetlo plavom označite strani ključ}
				
				\begin{longtabu} to \textwidth {|X[6, l]|X[6, l]|X[20, l]|}
					
					\hline \multicolumn{3}{|c|}{\textbf{korisnik - ime tablice}}	 \\[3pt] \hline
					\endfirsthead
					
					\hline \multicolumn{3}{|c|}{\textbf{korisnik - ime tablice}}	 \\[3pt] \hline
					\endhead
					
					\hline 
					\endlastfoot
					
					\cellcolor{LightGreen}IDKorisnik & INT	&  	Lorem ipsum dolor sit amet, consectetur adipiscing elit, sed do eiusmod tempor incididunt ut labore et dolore magna aliqua. Ut enim ad minim veniam 	\\ \hline
					korisnickoIme	& VARCHAR &   	\\ \hline 
					email & VARCHAR &   \\ \hline 
					ime & VARCHAR	&  		\\ \hline 
					\cellcolor{LightBlue} primjer	& VARCHAR &   	\\ \hline 
					
					
				\end{longtabu}
			
			
			\subsection{Dijagram baze podataka}
				\textit{ U ovom potpoglavlju potrebno je umetnuti dijagram baze podataka. Primarni i strani ključevi moraju biti označeni, a tablice povezane. Bazu podataka je potrebno normalizirati. Podsjetite se kolegija "Baze podataka".}
			
			\eject
			
			
		\section{Dijagram razreda}
		
			\textit{Potrebno je priložiti dijagram razreda s pripadajućim opisom. Zbog preglednosti je moguće dijagram razlomiti na više njih, ali moraju biti grupirani prema sličnim razinama apstrakcije i srodnim funkcionalnostima.}\\
			
			\textbf{\textit{dio 1. revizije}}\\
			
			\textit{Prilikom prve predaje projekta, potrebno je priložiti potpuno razrađen dijagram razreda vezan uz \textbf{generičku funkcionalnost} sustava. Ostale funkcionalnosti trebaju biti idejno razrađene u dijagramu sa sljedećim komponentama: nazivi razreda, nazivi metoda i vrste pristupa metodama (npr. javni, zaštićeni), nazivi atributa razreda, veze i odnosi između razreda.}\\
			
			\textbf{\textit{dio 2. revizije}}\\			
			
			\textit{Prilikom druge predaje projekta dijagram razreda i opisi moraju odgovarati stvarnom stanju implementacije}
			
			
			
			\eject
		
		\section{Dijagram stanja}
			
			
			\textbf{\textit{dio 2. revizije}}\\
			
			\textit{Potrebno je priložiti dijagram stanja i opisati ga. Dovoljan je jedan dijagram stanja koji prikazuje \textbf{značajan dio funkcionalnosti} sustava. Na primjer, stanja korisničkog sučelja i tijek korištenja neke ključne funkcionalnosti jesu značajan dio sustava, a registracija i prijava nisu. }
			
			
			\eject 
		
		\section{Dijagram aktivnosti}
			
			\textbf{\textit{dio 2. revizije}}\\
			
			 \textit{Potrebno je priložiti dijagram aktivnosti s pripadajućim opisom. Dijagram aktivnosti treba prikazivati značajan dio sustava.}
			
			\eject
		\section{Dijagram komponenti}
		
			\textbf{\textit{dio 2. revizije}}\\
		
			 \textit{Potrebno je priložiti dijagram komponenti s pripadajućim opisom. Dijagram komponenti treba prikazivati strukturu cijele aplikacije.}